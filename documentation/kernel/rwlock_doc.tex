\section{Reader/Writer Locks}

These locks are designed for locking critical code that is accessed in one of two ways, reading
some data or writing some data. For example, a read lock would be established to loop though a 
linked-list, but a write lock would be needed to actually modify the linked-list. These function provide
functionality for reader/writer locks.

There may be any number of reader locks acquired on the same lock, but only one writer. In addition, 
the writer may only be established if there are no readers, and a reader may only be established if
there is no write lock.

Internally, the lock state is maintained by an unsigned integer, where the 0th bit indicates the 
state of the write lock, and the rest of the bits are the read lock counts. If the 0th bit is 1, then
there is a writer lock established. Each time a reader lock is acquired, the lock state integer is
incremented by two (skips over the 0th bit), and each time a reader lock is release it is decremented
by two. This allows compact storage of the lock state information, and allows reducing the accesses of
the state integer to atomic operations.

An rwlock may not be acquired within an interrupt, unless section of code that is locked by that
lock in any way is surrounded by clearing interrupts and resetting them, where set\_int() is called
BEFORE the call to rwlock\_acquire, and the interrupts are restored AFTER the call to rwlock\_release.

\subsection{Definitions}
\begin{itemize}
\item \texttt{\textit{rwlock\_t}} 
\begin{lstlisting}
typedef volatile struct {
	volatile unsigned magic, flags;
	volatile unsigned long locks;
} rwlock_t;
\end{lstlisting}
\item \texttt{RWL\_READER}
\item \texttt{RWL\_WRITER}
\end{itemize}

% \lstset{language=C,caption={Descriptive Caption Text},label=DescriptiveLabel}

\subsection{Functions and Macros}

\begin{itemize}
\item \function{rwlock\_t *}{rwlock\_create}{\argument{rwlock\_t*}{rwl}} \\
Takes in a pointer to a \texttt{\textit{rwlock\_t}} structure (optional). If \texttt{rwl} is null, allocates
memory from kmalloc and returns that. Otherwise returns \texttt{rwl}. Sets up an \texttt{\textit{rwlock\_t}}
structure to be used as a reader/writer lock.

\item \function{void}{rwlock\_destroy}{\argument{rwlock\_t*}{rwl}} \\
Resets values inside \texttt{rwl}. If the lock is currently locked by anything,
this will cause a kernel panic. If \texttt{rwl} was allocated from \method{kmalloc} (from \method{rwlock\_create}(0)), this frees it.

\item \function{void}{rwlock\_acquire}{\argument{rwlock\_t*}{rwl}, \argument{int}{flags}} \\
Blocks until a lock of type specified by \texttt{flags} can be acquired. \texttt{flags} may
be \texttt{RWL\_READER} or \texttt{RWL\_WRITER}. 
\item \function{void}{rwlock\_escalate}{\argument{rwlock\_t*}{rwl}, \argument{int}{flags}} \\
Changes the lock from the current state to the state spcified by \texttt{flags}. Valid calls are either
the lock is a read lock, and \texttt{RWL\_WRITER} is passed or the lock is a write lock, and \texttt{RWL\_READER}
is passed. Only call this function if there is already an establish lock by the calling process.
\item \function{void}{rwlock\_release}{\argument{rwlock\_t*}{rwl}, \argument{int}{flags}} \\
Releases an acquired lock, of type specified by \texttt{flags}. The lock may be in either a reader or a writer state to call this, but
make sure to tell it to release the correct state! If a reader lock is establish and this is called
with flags as \texttt{RWL\_WRITER}, bad things will happen!
\end{itemize}

\subsection{Configuration}
\begin{itemize}
\item \texttt{CONFIG\_DEBUG} will cause the \method{rwlock\_acquire} and \method{rwlock\_escalate}
methods to track how long a task blocks for a lock, and if it times-out it will panic.
\end{itemize}

\subsection{Usage}
\begin{lstlisting}
#include <rwlock.h>

rwlock_t *rwl = rwl_create(0);
rwlock_acquire(rwl, RWL_READER);
/* reading stuff.... */
rwlock_escalate(rwl, RWL_WRITER);
/* write stuff */
rwlock_escalate(rwl, RWL_READER);
/* read stuff again */
rwlock_release(rwl, RWL_READER);

rwlock_destroy(rwl);
\end{lstlisting}
\subsection{Files}
\begin{itemize}
\item \texttt{include/rwlock.h}
\item \texttt{kernel/rwlock.c}
\end{itemize}
