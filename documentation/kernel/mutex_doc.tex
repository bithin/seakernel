\section{Mutexes}

Mutexes are locks that may either be locked or unlocked. They prevent sections of
critical code from being accessed by more than one process at a time.

\subsection{Definitions}
\begin{itemize}

\item \texttt{\textit{mutex\_t}} 
\begin{lstlisting}
typedef struct {
	unsigned magic; /* for debugging */
	unsigned lock; /* the state of the lock */
	unsigned flags;
	int pid; /* owner (-1 for unowned) */
} mutex_t;
\end{lstlisting}

\end{itemize}

\subsection{Functions and Macros}
\begin{itemize}
\item \function{mutex\_t *}{mutex\_create}{\argument{mutex\_t*}{mutex}} \\
Takes in a pointer to an already existing mutex (optional). If \texttt{mutex} is 
null, it returns a pointer to a kmalloc'd mutex. The returned pointer
is a pointer to a mutex that has been initialized so it is ready to use.

\item \function{void}{mutex\_destroy}{\argument{mutex\_t*}{mutex}} \\
Takes in a pointer to a mutex. Resets the flags, and if the mutex was allocated
with \method{kmalloc} (in \method{mutex\_create}), it frees it.
\item \function{void}{mutex\_acquire}{\argument{mutex\_t*}{mutex}} 
Blocks until the mutex can be acquired. Once this returns,
the calling process has exclusive access to the code block until \method{mutex\_release} is called.
Calling this in the same process again before releasing the mutex will cause a kernel panic.
\item \function{void}{mutex\_release}{\argument{mutex\_t*}{mutex}} 
Releases the mutex. Calling this without first acquiring the mutex will cause a kernel panic.
\end{itemize}
\subsection{Usage}
% \lstset{language=C,caption={Descriptive Caption Text},label=DescriptiveLabel}
\begin{lstlisting}
#include <mutex.h>

mutex_t mutex1;
mutex_t *mutex2;

mutex_create(&mutex1);
mutex2 = mutex_create(0);

mutex_acquire(&mutex1);
/* critical code */
mutex_release(&mutex1);

mutex_acquire(mutex2);
/* other critical code */
mutex_release(mutex2);

mutex_destroy(&mutex1);
mutex_destroy(mutex2);

\end{lstlisting}


\subsection{Files}
\begin{itemize}
\item \texttt{include/mutex.h}
\item \texttt{kernel/mutex.c}
\end{itemize}
