\section{Linked Lists}

Provides abstrated and uniform access to  a linked-list type data structure
across the kernel. These lists are kept internally as non-circular doubly 
linked-lists with a pointer to the head.

\subsection{Definitions}

\begin{itemize}
\item \texttt{\textit{struct llist}}
\begin{lstlisting}
struct llist {
	struct llistnode *head;
	rwlock_t rwl;
	char flags;
}
\end{lstlisting}

\item \texttt{\textit{struct llistnode}}
\begin{lstlisting}
struct llistnode {
	struct llistnode *next, *prev;
	void *entry;
}
\end{lstlisting}
\end{itemize}

\subsection{Functions and Macros}

\begin{itemize}
\item \function{struct llist *}{ll\_create}{\argument{struct llist*}{list}} \\
Takes in a pointer to an uninitialized list (optional). Returns a new list. 
If \texttt{list} is null this function will allocate a new list with 
\method{kmalloc} and return a pointer to that.

\item \function{void}{ll\_destroy}{\argument{struct llist*}{list}} \\
Takes in a pointer to an initialized and empty list, and resets the values
inside the structure. If the list was allocated with \method{kmalloc}, it is freed.
This function will panic if the list is not empty. 
\item \function{struct llistnode *}{ll\_insert}{\argument{struct llist*}{list}, \argument{void*}{entry}} \\
Inserts \texttt{entry} into \texttt{list}. The insert is done at the head of the list. 
Returns a pointer to the element of the list that contains the pointer to \texttt{entry}.
This is useful for removing the element later. This function will attempt to acquire a writer lock
on the list.

\item \function{void}{ll\_remove}{\argument{struct llist*}{list}, \argument{struct llistnode*}{node}} \\
Removes \texttt{node} from \texttt{list}. Note that this requires a pointer to the node that contains
the pointer to entry (as passed to \method{ll\_insert}), not the entry itself. This function will attempt to acquire 
a writer lock on the list.

\item \function{void}{ll\_remove\_element}{\argument{struct llist*}{list}, \argument{void*}{entry}} \\
Removes \texttt{entry} from \texttt{list}. Unlike \method{ll\_remove}, this function will search the
list for entry, and then remove it. This function will attempt to acquire a writer lock on the list.

\item \function{void *}{ll\_remove\_head}{\argument{struct llist*}{list}} \\
Removes the head entry in the list, and returns it. If the list is empty, it returns null.

\item \function{char}{ll\_is\_active}{\argument{struct llist*}{list}} \\
Returns true if the list is initialized and ready to be used. False otherwise.

\item \function{unsigned}{ll\_is\_empty}{\argument{struct llist*}{list}} \\
Returns true if the list is empty. False otherwise. This does not check if the list is initialized.

\item \macro{ll\_for\_each}{list,curnode} \\
This macro gets pre-processed into a for loop that loops though the list. On each iteration of the loop,
curnode points to the current node of the list. \texttt{list} should be of type \texttt{\textit{struct llist *}}
and \texttt{curnode} should be of type \texttt{\textit{struct llistnode *}} (this holds for all ll\_for macros).

\item \macro{ll\_for\_each\_entry}{list,curnode,type,entry} \\
Same as \method{ll\_for\_each}, except that the loop automatically gets the entry
for each node and puts it in \texttt{entry}. \texttt{type} denotes the type of the
pointer that entry is.

\item \macro{ll\_for\_each\_safe}{list,curnode,next} \\
Loops through the list, setting \texttt{curnode} to each node on each iteration, except that it
also backs up the next node for the next iteration so the list may be modified inside the loop.
If the list is modified inside the loop, it is important to call \method{ll\_maybe\_reset\_loop} before
the next iteration.

\item \macro{ll\_for\_each\_entry\_safe}{list,curnode,next,type,entry} \\
Same as \method{ll\_for\_each\_safe}, except that it also sets a \texttt{entry} to the
current entry of the list as well.

\item \macro{ll\_maybe\_reset\_loop}{list,curnode,next} \\
Looks to see if something happened in the loop that would require the loop to restart. This should be
called if the list is modified from inside the loop.
\end{itemize}

It is important to note three things about the loop macros:
\begin{enumerate}
\item The loop macros do NOT guarantee that each entry is presented only once. If the list is modified
while the loop is running, the same entry may show up twice. Editing the list from inside a loop is not safe.
Special care must be taken when doing this.

\item The loop macros do NOT acquire any locks on the list while running. To use the loop macros, you should
first acquire at least a reader lock on the entire list.

\item After the loop terminates, the values of curnode, next, and entry may not be zero - in fact, 
their values outside the loop are undefined by the loop macros.
\end{enumerate}

\subsection{Files}
\begin{itemize}
\item \texttt{include/ll.h}
\item \texttt{library/klib/ll.c}
\end{itemize}
